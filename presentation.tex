\documentclass{beamer}
\usepackage[utf8]{inputenc}

\title{Présentation Merise}
\author{Fougerolle, Requena, Tabbara}
\date{Février 2019}

\usepackage{natbib}
\usepackage{graphicx}

\begin{document}

\maketitle
\begin{frame}
\frametitle{Merise}
\begin{itemize}
\item René Colletti, Arnold Rochfeld et Hubert Tardie
\item Années 197
\item Méthode d'analyse, de gestion et de conception de projet informatiqu
\item Projet opérationnel début des années 1980 à la demande du ministère de l'industri
\item Utilisé en France dans les administrations publiques ou privées
\end{itemize}
\end{frame}
\begin{frame}
\frametitle{MCD}
\framesubtitle{modèle conceptuel des donnée}
\begin{itemize}
\item Notion d'entité et d'association
\item Notion de relation
\item Décrire la sémantique du domaine
\end{itemize}
\end{frame}
\begin{frame}
\frametitle{MLD}
\framesubtitle{modèle logique des données}
Adapté à une implémentation ultérieure :
\begin{itemize}
\item Niveau physique
\item Sous forme de base de données relationnelle ou réseau
\end{itemize}
\end{frame}
\begin{frame}
\frametitle{MOT}
\framesubtitle{modèle organisationnel des traitements} 
Représentation de l'activité de l'organisme étudié :
\begin{itemize}
\item Traitements entre l'homme et la machine \item Période de déroulement de chaque tâche
\item Répartition de la responsabilité de ces traitements (tâches) au niveau des microstructures : services, départements, divisions, poste de travail, bureaux, …
\end{itemize}
\end{frame}

\end{document}
